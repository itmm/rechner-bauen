\documentclass[a5paper,ngerman]{article}
\usepackage[T1]{fontenc}
\usepackage[utf8]{inputenc}
\usepackage[margin=0.5in,includefoot]{geometry}
\usepackage{microtype}
\usepackage{babel}
\usepackage{siunitx}
\usepackage[light]{solarized}
\usepackage{sectsty}
\usepackage{ccfonts}
\usepackage{euler}
\usepackage{fancyhdr}
\usepackage{fancyvrb}
\usepackage{graphicx}
\usepackage{multicol}
\usepackage{mdframed}
\usepackage{lisp}
\pagestyle{fancy}
\fancypagestyle{plain}{
\fancyhf{}
\fancyfoot[C]{{\color{deemph}\small$\thepage$}}
\renewcommand{\headrulewidth}{0pt}
\renewcommand{\footrulewidth}{0pt}}
\title{\color{emph}Warum einen Rechner bauen?}
\author{Timm Knape}
\date{\today}
\columnseprule.2pt
\renewcommand{\columnseprulecolor}{\color{deemph}}
\begin{document}
\pagecolor{background}
\color{normal}
\allsectionsfont{\color{emph}\mdseries}
\pagestyle{plain}
\maketitle
\thispagestyle{fancy}
\surroundwithmdframed[backgroundcolor=codebackground,fontcolor=normal,hidealllines=true]{Verbatim}

Mal im Ernst:
Warum sollte man seine eigenen Rechner bauen?
Gibt es nicht schon genug Rechner?
Für wenige hundert Euro bekommt man eine ganz brauchbare tragbare
Ausführung.
Machen wir uns nichts vor:
Preiswerter können wir einen solches Gerät nicht fertigen.
Durch Skalen-Effekte macht es überhaupt keinen Sinn,
mit einer Einzelfertigung mit den großen Herstellern in Fernost
in Konkurrenz treten zu wollen.

Aber mich treibt etwas anderes an.
Ich bin fest davon überzeugt,
dass ich ein viel besserer Programmierer werde,
wenn ich genau weiß, wie mein Arbeitsgerät funktioniert.
Und es macht sehr viel Spass.

Zugegeben: ein bisschen wird man dabei zum verrückten Professor.
Belächelt von der Gesellschaft.
Einen eigenen Rechner zu bauen, dass ist so skurril wie:

\begin{itemize}
\item sein eigenes Bier brauen,
\item oder seine eigene Musik spielen,
\item oder seine eigenen Programme schreiben,
\item oder seine eigene Steuererklärung machen,
\item oder sein eigenes Leben leben.
\end{itemize}

Ich hoffe, ihr merkt, dass an irgend einer Stelle es gar nicht mehr so
irrsinnig wirkt.

Wir müssen ja nicht gleich einen Super-Rechner bauen.
Wir fangen ganz einfach an.

\section{Voraussetzungen}

Ich bin selber kein Held am Lötkolben.

Ich bin nicht reich.

Und ich bin ungeduldig.

Aber ich denke, ich bin stur.

Sprich ich denke, jeder kann mit mir zusammen einen Rechner bauen,
wenn er nur stur genug ist.

\subsection{Welches Betriebssystem brauche ich?}

Ich persönlich probiere die großen Betriebssysteme Windows und macOS so
weit wie möglich zu meiden.
Diesen Text erstelle ich (noch) auf einem Rasperry Pi mit einer
Linux-Distribution.
Aber ich bin mir bewußt, dass ich so eine Ausstattung nicht voraussetzen
kann.

Daher habe ich mich zu einem sehr gewagten Schritt entschlossen:
Es ist völlig egal, welchen Rechner ihr habt.
Zum erstellen des Rechners brauchen wir nur Rechner, die wir selbst
gebaut haben.
Für die ersten Rechner brauchen wir gar keine Rechner.

Höchstens um die Bauteile einzukaufen und diese Anleitung zu lesen.
Ein Zugriff auf das Internet mit all seinen Versuchungen ist also
hilfreich, aber nicht notwendig.

\subsection{Was für einen Rechner bauen wir?}

Der erste Rechner wird sehr einfach sein.

Stellt euch einen minimalen Rechner vor.
Unsere erste Version wird noch viel einfacher sein.

Ganz grob orientieren wir uns an den Altair 8800.
Ich habe ihm den Arbeitstitel Alder-80 gegeben.

Es gibt erstmal keine Tastatur.
Keinen Bildschirm.
Nur jede Menge Schalter und LEDs.

Aber er ist ein Rechner.
Wir können ihn mit den Schaltern programmieren.
Das Programm ausführen.
Und uns danach die Ergebnisse ansehen.

Das alles mit zwei integrierten Bausteinen,
einer handvoll Schalter und Widerstände.
Und ein paar Experimentier-Steckboards und einem USB-Ladegerät.

\subsection{Ausblick}

Diesen Rechner können wir langsam erweitern,
bis er die Funktionalität eines normalen Rechners hat.

Der Weg ist lang.
Der Weg ist steinig.
Aber wir müssen den Weg selber gehen.
Nur so entsteht Erkenntnis.

Wie heißt es so schön:
Man wird nicht sportlich, indem man anderen Leuten im Fitness-Studio
zusieht.

Also weg mit der Berieselung.
Selbst ist der Nerd.

Lasset die Spiele beginnen.

\end{document}
